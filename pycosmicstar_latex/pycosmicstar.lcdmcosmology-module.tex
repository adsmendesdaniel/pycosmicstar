%
% API Documentation for API Documentation
% Module pycosmicstar.lcdmcosmology
%
% Generated by epydoc 3.0.1
% [Mon Aug 11 14:45:22 2014]
%

%%%%%%%%%%%%%%%%%%%%%%%%%%%%%%%%%%%%%%%%%%%%%%%%%%%%%%%%%%%%%%%%%%%%%%%%%%%
%%                          Module Description                           %%
%%%%%%%%%%%%%%%%%%%%%%%%%%%%%%%%%%%%%%%%%%%%%%%%%%%%%%%%%%%%%%%%%%%%%%%%%%%

    \index{pycosmicstar \textit{(package)}!pycosmicstar.lcdmcosmology \textit{(module)}|(}
\section{Module pycosmicstar.lcdmcosmology}

    \label{pycosmicstar:lcdmcosmology}
\textbf{Version:} 1.0.1



\textbf{Author:} Eduardo dos Santos Pereira



\textbf{License:} GPLV3




%%%%%%%%%%%%%%%%%%%%%%%%%%%%%%%%%%%%%%%%%%%%%%%%%%%%%%%%%%%%%%%%%%%%%%%%%%%
%%                               Variables                               %%
%%%%%%%%%%%%%%%%%%%%%%%%%%%%%%%%%%%%%%%%%%%%%%%%%%%%%%%%%%%%%%%%%%%%%%%%%%%

  \subsection{Variables}

    \vspace{-1cm}
\hspace{\varindent}\begin{longtable}{|p{\varnamewidth}|p{\vardescrwidth}|l}
\cline{1-2}
\cline{1-2} \centering \textbf{Name} & \centering \textbf{Description}& \\
\cline{1-2}
\endhead\cline{1-2}\multicolumn{3}{r}{\small\textit{continued on next page}}\\\endfoot\cline{1-2}
\endlastfoot\raggedright \_\-\_\-e\-m\-a\-i\-l\-\_\-\_\- & \raggedright \textbf{Value:} 
{\tt \texttt{'}\texttt{pereira.somoza@gmail.com}\texttt{'}}&\\
\cline{1-2}
\raggedright \_\-\_\-c\-r\-e\-d\-i\-t\-s\-\_\-\_\- & \raggedright \textbf{Value:} 
{\tt \texttt{[}\texttt{'}\texttt{Eduardo dos Santos Pereira}\texttt{'}\texttt{]}}&\\
\cline{1-2}
\raggedright \_\-\_\-m\-a\-i\-n\-t\-a\-i\-n\-e\-r\-\_\-\_\- & \raggedright \textbf{Value:} 
{\tt \texttt{'}\texttt{Eduardo dos Santos Pereira}\texttt{'}}&\\
\cline{1-2}
\raggedright \_\-\_\-s\-t\-a\-t\-u\-s\-\_\-\_\- & \raggedright The Cold Dark Matter plus Cosmological constant Module (LCDM)

          Na atual versao usamos a normalizacao do WMAP (sem ondas 
          gravitacionais) a expressao foi adaptada de Eisenstein e Hu (ApJ 
          511, 5, 1999) de forma a fornecer sigma\_8 = 0,84. A fracao de 
          massa dos halos e obtida de Sheth e Tormen (MNRAS 308, 119, 1999)
          Todos os modelos consideram Omega\_Total = Omega\_M + Omega\_L = 
          1,0

          "Best Fit" do WMAP-3: omega\_m = 0,238, omega\_b = 0,042, 
          omega\_l = 0,762, h = 0,734, sigma\_8 = 0,744 Veja que sigma\_8 
          pelo WMAP e' obtido atraves da recombinacao. Outras estimativas 
          (p.e. aglomerados de galaxias) fornecem sigma\_8 = 0,84. Conjunto
          de dados: WMAP-3: omega\_m = 0,238, omega\_b = 0,042, omega\_l = 
          0,762 h = 0,734, sigma\_8 = 0,84 WMAP-1: omega\_m = 0,29, 
          omega\_b = 0,44, omega\_l = 0,71 h = 0,72, sigma\_8 = 0,9

          This file is part of pystar. copyright : Eduardo dos Santos 
          Pereira

          pystar is free software: you can redistribute it and/or modify it
          under the terms of the GNU General Public License as published by
          the Free Software Foundation, either version 3 of the License. 
          pystar is distributed in the hope that it will be useful, but 
          WITHOUT ANY WARRANTY; without even the implied warranty of 
          MERCHANTABILITY or FITNESS FOR A PARTICULAR PURPOSE.  See the GNU
          General Public License for more details.

          You should have received a copy of the GNU General Public License
          along with Foobar.  If not, see 
          {\textless}http://www.gnu.org/licenses/{\textgreater}.

\textbf{Value:} 
{\tt \texttt{'}\texttt{Stable}\texttt{'}}&\\
\cline{1-2}
\raggedright c\-o\-s\-m\-o\-l\-i\-b\-I\-m\-p\-o\-r\-t\-S\-t\-a\-t\-u\-s\- & \raggedright \textbf{Value:} 
{\tt True}&\\
\cline{1-2}
\raggedright \_\-\_\-p\-a\-c\-k\-a\-g\-e\-\_\-\_\- & \raggedright \textbf{Value:} 
{\tt \texttt{'}\texttt{pycosmicstar}\texttt{'}}&\\
\cline{1-2}
\end{longtable}


%%%%%%%%%%%%%%%%%%%%%%%%%%%%%%%%%%%%%%%%%%%%%%%%%%%%%%%%%%%%%%%%%%%%%%%%%%%
%%                           Class Description                           %%
%%%%%%%%%%%%%%%%%%%%%%%%%%%%%%%%%%%%%%%%%%%%%%%%%%%%%%%%%%%%%%%%%%%%%%%%%%%

    \index{pycosmicstar \textit{(package)}!pycosmicstar.lcdmcosmology \textit{(module)}!pycosmicstar.lcdmcosmology.lcdmcosmology \textit{(class)}|(}
\subsection{Class lcdmcosmology}

    \label{pycosmicstar:lcdmcosmology:lcdmcosmology}
\begin{tabular}{cccccc}
% Line for pycosmicstar.cosmology.cosmology, linespec=[False]
\multicolumn{2}{r}{\settowidth{\BCL}{pycosmicstar.cosmology.cosmology}\multirow{2}{\BCL}{pycosmicstar.cosmology.cosmology}}
&&
  \\\cline{3-3}
  &&\multicolumn{1}{c|}{}
&&
  \\
&&\multicolumn{2}{l}{\textbf{pycosmicstar.lcdmcosmology.lcdmcosmology}}
\end{tabular}

\begin{alltt}
The Cold Dark Matter (CDM) plus Cosmolocical Constan (Lambda) -  lcdm

Keyword arguments:
    omegam -- (default 0.24) - The dark matter parameter

    omegab -- (default 0.04) - The barionic parameter

    omegal -- (default 0.73) - The dark energy parameter

    h -- (default 0.7) - The h of the Hubble constant (H = h * 100)
\end{alltt}


%%%%%%%%%%%%%%%%%%%%%%%%%%%%%%%%%%%%%%%%%%%%%%%%%%%%%%%%%%%%%%%%%%%%%%%%%%%
%%                                Methods                                %%
%%%%%%%%%%%%%%%%%%%%%%%%%%%%%%%%%%%%%%%%%%%%%%%%%%%%%%%%%%%%%%%%%%%%%%%%%%%

  \subsubsection{Methods}

    \label{pycosmicstar:lcdmcosmology:lcdmcosmology:__init__}
    \index{pycosmicstar \textit{(package)}!pycosmicstar.lcdmcosmology \textit{(module)}!pycosmicstar.lcdmcosmology.lcdmcosmology \textit{(class)}!pycosmicstar.lcdmcosmology.lcdmcosmology.\_\_init\_\_ \textit{(method)}}

    \vspace{0.5ex}

\hspace{.8\funcindent}\begin{boxedminipage}{\funcwidth}

    \raggedright \textbf{\_\_init\_\_}(\textit{self}, \textit{omegam}={\tt 0.24}, \textit{omegab}={\tt 0.04}, \textit{omegal}={\tt 0.73}, \textit{h}={\tt 0.7})

\setlength{\parskip}{2ex}
\setlength{\parskip}{1ex}
    \end{boxedminipage}

    \vspace{0.5ex}

\hspace{.8\funcindent}\begin{boxedminipage}{\funcwidth}

    \raggedright \textbf{dt\_dz}(\textit{self}, \textit{z})

\setlength{\parskip}{2ex}
    Return the relation between the cosmic time and the redshift

\setlength{\parskip}{1ex}
      Overrides: pycosmicstar.cosmology.cosmology.dt\_dz 	extit{(inherited documentation)}

    \end{boxedminipage}

    \vspace{0.5ex}

\hspace{.8\funcindent}\begin{boxedminipage}{\funcwidth}

    \raggedright \textbf{dr\_dz}(\textit{self}, \textit{z})

\setlength{\parskip}{2ex}
    Return the comove-radii

\setlength{\parskip}{1ex}
      Overrides: pycosmicstar.cosmology.cosmology.dr\_dz 	extit{(inherited documentation)}

    \end{boxedminipage}

    \vspace{0.5ex}

\hspace{.8\funcindent}\begin{boxedminipage}{\funcwidth}

    \raggedright \textbf{H}(\textit{self}, \textit{z})

    \vspace{-1.5ex}

    \rule{\textwidth}{0.5\fboxrule}
\setlength{\parskip}{2ex}
\begin{alltt}
Return the Hubble parameter as a function of z.

Keyword arguments:
    z -- redshift
\end{alltt}

\setlength{\parskip}{1ex}
      Overrides: pycosmicstar.cosmology.cosmology.H

    \end{boxedminipage}

    \vspace{0.5ex}

\hspace{.8\funcindent}\begin{boxedminipage}{\funcwidth}

    \raggedright \textbf{dV\_dz}(\textit{self}, \textit{z})

    \vspace{-1.5ex}

    \rule{\textwidth}{0.5\fboxrule}
\setlength{\parskip}{2ex}
\begin{alltt}
Return the comove volume variation.

Keyword arguments:
    z -- redshift
\end{alltt}

\setlength{\parskip}{1ex}
      Overrides: pycosmicstar.cosmology.cosmology.dV\_dz

    \end{boxedminipage}

    \vspace{0.5ex}

\hspace{.8\funcindent}\begin{boxedminipage}{\funcwidth}

    \raggedright \textbf{dgrowth\_dt}(\textit{self}, \textit{z})

    \vspace{-1.5ex}

    \rule{\textwidth}{0.5\fboxrule}
\setlength{\parskip}{2ex}
\begin{alltt}
Return the derivative of the growth function with
respect to  time.

Keyword arguments:
    z -- redshift
\end{alltt}

\setlength{\parskip}{1ex}
      Overrides: pycosmicstar.cosmology.cosmology.dgrowth\_dt

    \end{boxedminipage}

    \vspace{0.5ex}

\hspace{.8\funcindent}\begin{boxedminipage}{\funcwidth}

    \raggedright \textbf{growthFunction}(\textit{self}, \textit{z})

    \vspace{-1.5ex}

    \rule{\textwidth}{0.5\fboxrule}
\setlength{\parskip}{2ex}
\begin{alltt}
Return the growth function

Keyword arguments:
    z -- redshift
\end{alltt}

\setlength{\parskip}{1ex}
      Overrides: pycosmicstar.cosmology.cosmology.growthFunction

    \end{boxedminipage}

    \vspace{0.5ex}

\hspace{.8\funcindent}\begin{boxedminipage}{\funcwidth}

    \raggedright \textbf{sigma}(\textit{self}, \textit{kmass})

    \vspace{-1.5ex}

    \rule{\textwidth}{0.5\fboxrule}
\setlength{\parskip}{2ex}
\begin{alltt}
Return the sigma.

Keyword arguments:
    kmass -- mass scale
\end{alltt}

\setlength{\parskip}{1ex}
      Overrides: pycosmicstar.cosmology.cosmology.sigma

    \end{boxedminipage}

    \vspace{0.5ex}

\hspace{.8\funcindent}\begin{boxedminipage}{\funcwidth}

    \raggedright \textbf{dsigma2\_dk}(\textit{self}, \textit{kl})

    \vspace{-1.5ex}

    \rule{\textwidth}{0.5\fboxrule}
\setlength{\parskip}{2ex}
    "Return the integrating of sigma(M,z) for a top-hat filtering. In z = 0
    return sigma\_8, for z {\textgreater} 0 return sigma(M,z)

\setlength{\parskip}{1ex}
      Overrides: pycosmicstar.cosmology.cosmology.dsigma2\_dk

    \end{boxedminipage}

    \vspace{0.5ex}

\hspace{.8\funcindent}\begin{boxedminipage}{\funcwidth}

    \raggedright \textbf{rodm}(\textit{self}, \textit{z})

    \vspace{-1.5ex}

    \rule{\textwidth}{0.5\fboxrule}
\setlength{\parskip}{2ex}
\begin{alltt}
Return the dark matter density

Keyword arguments:
    z -- redshift
\end{alltt}

\setlength{\parskip}{1ex}
      Overrides: pycosmicstar.cosmology.cosmology.rodm

    \end{boxedminipage}

    \vspace{0.5ex}

\hspace{.8\funcindent}\begin{boxedminipage}{\funcwidth}

    \raggedright \textbf{robr}(\textit{self}, \textit{z})

    \vspace{-1.5ex}

    \rule{\textwidth}{0.5\fboxrule}
\setlength{\parskip}{2ex}
\begin{alltt}
Return the barionic density.

Keyword arguments:
    z -- redshift
\end{alltt}

\setlength{\parskip}{1ex}
      Overrides: pycosmicstar.cosmology.cosmology.robr

    \end{boxedminipage}

    \vspace{0.5ex}

\hspace{.8\funcindent}\begin{boxedminipage}{\funcwidth}

    \raggedright \textbf{age}(\textit{self}, \textit{z})

    \vspace{-1.5ex}

    \rule{\textwidth}{0.5\fboxrule}
\setlength{\parskip}{2ex}
\begin{alltt}
Return the age of the Universe for some redshift.

Keyword arguments:
    z -- redshift
\end{alltt}

\setlength{\parskip}{1ex}
      Overrides: pycosmicstar.cosmology.cosmology.age

    \end{boxedminipage}

    \vspace{0.5ex}

\hspace{.8\funcindent}\begin{boxedminipage}{\funcwidth}

    \raggedright \textbf{setCosmologicalParameter}(\textit{self}, \textit{omegam}, \textit{omegab}, \textit{omegal}, \textit{h})

    \vspace{-1.5ex}

    \rule{\textwidth}{0.5\fboxrule}
\setlength{\parskip}{2ex}
    Set the cosmological parameters

\setlength{\parskip}{1ex}
      Overrides: pycosmicstar.cosmology.cosmology.setCosmologicalParameter

    \end{boxedminipage}

    \vspace{0.5ex}

\hspace{.8\funcindent}\begin{boxedminipage}{\funcwidth}

    \raggedright \textbf{getCosmologicalParameter}(\textit{self})

    \vspace{-1.5ex}

    \rule{\textwidth}{0.5\fboxrule}
\setlength{\parskip}{2ex}
    Return the cosmological parameter

\setlength{\parskip}{1ex}
      Overrides: pycosmicstar.cosmology.cosmology.getCosmologicalParameter

    \end{boxedminipage}

    \label{pycosmicstar:lcdmcosmology:lcdmcosmology:getDeltaC}
    \index{pycosmicstar \textit{(package)}!pycosmicstar.lcdmcosmology \textit{(module)}!pycosmicstar.lcdmcosmology.lcdmcosmology \textit{(class)}!pycosmicstar.lcdmcosmology.lcdmcosmology.getDeltaC \textit{(method)}}

    \vspace{0.5ex}

\hspace{.8\funcindent}\begin{boxedminipage}{\funcwidth}

    \raggedright \textbf{getDeltaC}(\textit{self})

    \vspace{-1.5ex}

    \rule{\textwidth}{0.5\fboxrule}
\setlength{\parskip}{2ex}
    Return the critical density

\setlength{\parskip}{1ex}
    \end{boxedminipage}

    \vspace{0.5ex}

\hspace{.8\funcindent}\begin{boxedminipage}{\funcwidth}

    \raggedright \textbf{getTilt}(\textit{self})

\setlength{\parskip}{2ex}
\setlength{\parskip}{1ex}
      Overrides: pycosmicstar.cosmology.cosmology.getTilt

    \end{boxedminipage}

    \vspace{0.5ex}

\hspace{.8\funcindent}\begin{boxedminipage}{\funcwidth}

    \raggedright \textbf{getRobr0}(\textit{self})

    \vspace{-1.5ex}

    \rule{\textwidth}{0.5\fboxrule}
\setlength{\parskip}{2ex}
    Return the barionic matter density at the present day.

\setlength{\parskip}{1ex}
      Overrides: pycosmicstar.cosmology.cosmology.getRobr0

    \end{boxedminipage}

    \vspace{0.5ex}

\hspace{.8\funcindent}\begin{boxedminipage}{\funcwidth}

    \raggedright \textbf{getRodm0}(\textit{self})

    \vspace{-1.5ex}

    \rule{\textwidth}{0.5\fboxrule}
\setlength{\parskip}{2ex}
    Return the dark matter density at the present day.

\setlength{\parskip}{1ex}
      Overrides: pycosmicstar.cosmology.cosmology.getRodm0

    \end{boxedminipage}

    \index{pycosmicstar \textit{(package)}!pycosmicstar.lcdmcosmology \textit{(module)}!pycosmicstar.lcdmcosmology.lcdmcosmology \textit{(class)}|)}
    \index{pycosmicstar \textit{(package)}!pycosmicstar.lcdmcosmology \textit{(module)}|)}
